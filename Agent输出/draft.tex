```latex
\documentclass[10pt,twocolumn,letterpaper]{article}

% Required packages for CVPR template
\usepackage{graphicx}
\usepackage{amsmath}
\usepackage{amssymb}
\usepackage{natbib}
\usepackage{times}
\usepackage{epsfig}
\usepackage{psfrag}
\usepackage{algorithm}
\usepackage{algorithmic}
\usepackage{array}
\usepackage{url}

\begin{document}

\title{AutoTeX Research Paper}

\author{
Alice Zhang\\
Tsinghua University
\and
Bob Li\\
MIT
\and
Carol Chen\\
HKUST
}

\maketitle

\begin{abstract}
This document demonstrates the AutoTeX parsing flow.
\end{abstract}

\keywords{AutoTeX, Parsing, Schema, Word, Pipeline, 1 Introduction}

\section{1 Introduction}

This paper introduces the AutoTeX pipeline that converts complex documents into structured JSON outputs for downstream agents. The system leverages AI techniques and strict schema validation \cite{1}.

\subsection{1.1 Background}

Traditional document digitization workflows rely on manual formatting. AutoTeX automates this process by combining deterministic parsing with LLM-based inference.

$$E = mc^2$$

\begin{verbatim}
for i in range(3):
    print('AutoTeX rocks!')
\end{verbatim}

\begin{table}
\centering
\caption{Table 1: System Metrics}
\begin{tabular}{|c|c|c|}
\hline
Module & Latency (ms) & Accuracy \\
\hline
Parser & 32 & 0.91 \\
\hline
OCR & 58 & 0.87 \\
\hline
Renderer & 20 & 0.95 \\
\hline
\end{tabular}
\end{table}

Figure 1: AutoTeX Architecture

Further explanation with another paragraph referencing \cite{2}.

\bibliographystyle{plainnat}
\begin{thebibliography}{9}

\bibitem[1]{1} Alice Zhang, Bob Li. AutoTeX System Overview. Journal of Automation, 2024.

\bibitem[2]{2} Carol Chen. Structured Document Understanding. Conference on AI Systems, 2025.

\end{thebibliography}

\end{document}
```
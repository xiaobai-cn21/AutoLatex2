```latex
\documentclass[10pt,twocolumn,letterpaper]{article}

\usepackage{cvpr}
\usepackage{times}
\usepackage{epsfig}
\usepackage{graphicx}
\usepackage{amsmath}
\usepackage{amssymb}
\usepackage{subfigure}
\usepackage{url}
\usepackage{booktabs}
\usepackage[pagebackref=true,breaklinks=true,colorlinks,linkcolor={red},citecolor={red},urlcolor={red}]{hyperref}

\cvprfinalcopy

\begin{document}

\title{AutoTeX Research Paper}

\author{
Alice Zhang\\
Tsinghua University\\
\and
Bob Li\\
MIT\\
\and
Carol Chen\\
HKUST
}

\maketitle

\begin{abstract}
This document demonstrates the AutoTeX parsing flow.
\end{abstract}

\keywords{AutoTeX, Parsing, Schema, Word, Pipeline}

\section{Introduction}

This paper introduces the AutoTeX pipeline that converts complex documents into structured JSON outputs for downstream agents. The system leverages AI techniques and strict schema validation \cite{1}.

\subsection{Background}

Traditional document digitization workflows rely on manual formatting. AutoTeX automates this process by combining deterministic parsing with LLM-based inference.

\begin{equation}
E = mc^2
\end{equation}

\begin{verbatim}
for i in range(3):
    print('AutoTeX rocks!')
\end{verbatim}

\begin{table}[ht]
\centering
\caption{System Metrics}
\begin{tabular}{lcc}
\toprule
Module & Latency (ms) & Accuracy \\
\midrule
Parser & 32 & 0.91 \\
OCR & 58 & 0.87 \\
Renderer & 20 & 0.95 \\
\bottomrule
\end{tabular}
\end{table}

Figure 1: AutoTeX Architecture

Further explanation with another paragraph referencing \cite{2}.

\bibliographystyle{unsrt}
\bibliography{references}

\begin{thebibliography}{1}

\bibitem{1}
Alice Zhang, Bob Li.
\newblock AutoTeX System Overview.
\newblock Journal of Automation, 2024.

\bibitem{2}
Carol Chen.
\newblock Structured Document Understanding.
\newblock Conference on AI Systems, 2025.

\end{thebibliography}

\end{document}
```
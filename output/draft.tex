```latex
\documentclass{ieeeaccess}
\usepackage{cite}
\usepackage{amsmath,amssymb,amsfonts}
\usepackage{algorithmic}
\usepackage{graphicx}
\usepackage{textcomp}
\def\BibTeX{{\rm B\kern-.05em{\sc i\kern-.025em b}\kern-.08em
    T\kern-.1667em\lower.7ex\hbox{E}\kern-.125emX}}

\begin{document}
\history{Date of publication xxxx 00, 0000, date of current version xxxx 00, 0000.}
\doi{10.1109/ACCESS.2024.DOI}

\title{AutoTeX Research Paper}

\author{\uppercase{Alice Zhang}\authorrefmark{1},
\uppercase{Bob Li}\authorrefmark{2},
and \uppercase{Carol Chen}\authorrefmark{3}}

\address[1]{Tsinghua University, Beijing, China}
\address[2]{MIT, Cambridge, MA, USA}
\address[3]{HKUST, Hong Kong, China}

\corresp{Corresponding author: Alice Zhang (e-mail: alice.zhang@example.com).}

\begin{abstract}
This document demonstrates the AutoTeX parsing flow.
\end{abstract}

\begin{keywords}
AutoTeX, Parsing, Schema, Word, Pipeline
\end{keywords}

\titlepgskip=-15pt
\maketitle

\section{Introduction}
This paper introduces the AutoTeX pipeline that converts complex documents into structured JSON outputs for downstream agents. The system leverages AI techniques and strict schema validation \cite{cite1}.

\subsection{Background}
Traditional document digitization workflows rely on manual formatting. AutoTeX automates this process by combining deterministic parsing with LLM-based inference.

\begin{equation}
E = mc^2
\end{equation}

\begin{verbatim}
for i in range(3):
    print('AutoTeX rocks!')
\end{verbatim}

\begin{table}[htbp]
\centering
\caption{Table 1: System Metrics}
\begin{tabular}{|l|c|c|}
\hline
Module & Latency (ms) & Accuracy \\
\hline
Parser & 32 & 0.91 \\
OCR & 58 & 0.87 \\
Renderer & 20 & 0.95 \\
\hline
\end{tabular}
\end{table}

\begin{figure}[htbp]
\centering
% \includegraphics[width=\linewidth]{}
\caption{AutoTeX Architecture}
\end{figure}

Further explanation with another paragraph referencing \cite{cite2}.

\begin{thebibliography}{00}
\bibitem{cite1}
Alice Zhang, Bob Li. AutoTeX System Overview. Journal of Automation, 2024.

\bibitem{cite2}
Carol Chen. Structured Document Understanding. Conference on AI Systems, 2025.
\end{thebibliography}

\end{document}
```